\PassOptionsToPackage{top=3cm,left=3cm,right=3cm,bottom=3cm}{geometry}
\documentclass[fleqn,11pt]{wlscirep_supp}

\usepackage[]{minitoc}
\mtcsetdepth{secttoc}{3}
\setcounter{secnumdepth}{2}
\setcounter{tocdepth}{2}
\mtcsettitle{secttoc}{}


\usepackage[utf8]{inputenc}
\usepackage[T1]{fontenc}
\usepackage[english]{babel}
%\usepackage[top=3cm,left=3cm,right=3cm,bottom=3cm]{geometry}% by courtesy of Mico
\usepackage{lmodern}
\usepackage{bbm}
\usepackage{graphicx}
\usepackage{epstopdf}
\usepackage{colortbl}
\usepackage{siunitx}
\sisetup{
  detect-all,
  detect-weight=true,
  detect-family=true,
  mode=text,
%   detect-inline-family=math,
  group-separator={,},
%   group-minimum-digits={3}			
}
\usepackage{rotating}
\usepackage{tabularx}
\usepackage{tabu}
\usepackage{authblk}
\usepackage{mathtools}
\usepackage{overpic}
\usepackage{url}
\usepackage{tikz}
\usetikzlibrary{positioning}
\usetikzlibrary{arrows}
\usetikzlibrary{fit}
\usepackage{multirow}
\usepackage{float}
\usepackage[normalem]{ulem}
\usepackage{bm}
\usepackage{enumerate}
\usepackage[absolute,overlay%,showboxes
                        ]{textpos}
% \usepackage{caption}
\usepackage[font=small,labelfont=bf,justification=justified]{caption}
\usepackage{subcaption}
\usepackage{xspace}
\usepackage[colorinlistoftodos]{todonotes}
\usepackage{placeins}
\usepackage{makecell, booktabs}
\usepackage{eqparbox}
\usepackage{rotating}
\usepackage{graphicx}
\usepackage{xspace}
\usepackage{setspace}
%\usepackage{comment}
\usepackage[resetlabels,labeled]{multibib}
\newcites{Supp}{References}
\usepackage[sort&compress]{cleveref}
\Crefname{appendix}{Supplement}{Supplements}

%\usepackage{mathabx}

% Table formatting packages
\usepackage{dcolumn} % align decimal points in tables
\newcolumntype{d}[1]{D{.}{.}{#1}}
\usepackage{booktabs} 
\usepackage[flushleft]{threeparttable}
\usepackage{siunitx} % align on decimal point in tables
\usepackage{lineno}
\usepackage{etoolbox}

\usepackage{lscape}
\usepackage{longtable}
\usepackage{arydshln}

% math
\usepackage{amsmath,amsfonts,amssymb}
% additional math symbols
\DeclareFontFamily{U}{mathb}{}
\DeclareFontShape{U}{mathb}{m}{n}{
  <-5.5> mathb5
  <5.5-6.5> mathb6
  <6.5-7.5> mathb7
  <7.5-8.5> mathb8
  <8.5-9.5> mathb9
  <9.5-11.5> mathb10
  <11.5-> mathb12
}{}
\DeclareSymbolFont{mathb}{U}{mathb}{m}{n}
\DeclareMathSymbol{\ulsh}{3}{mathb}{"E8}
\DeclareMathSymbol{\ursh}{3}{mathb}{"E9}
\DeclareMathSymbol{\dlsh}{3}{mathb}{"EA}
\DeclareMathSymbol{\drsh}{3}{mathb}{"EB}

%% Patch 'normal' math environments:
\newcommand*\linenomathpatch[1]{%
  \cspreto{#1}{\linenomath}%
  \cspreto{#1*}{\linenomath}%
  \csappto{end#1}{\endlinenomath}%
  \csappto{end#1*}{\endlinenomath}%
}

\linenomathpatch{equation}
\linenomathpatch{gather}
\linenomathpatch{multline}
\linenomathpatch{align}
\linenomathpatch{alignat}
\linenomathpatch{flalign}

\linenumbers

% PLOS formatting
\makeatletter %only needed in preamble
\renewcommand\Large{\@setfontsize\Large{18pt}{18}}
\renewcommand\large{\@setfontsize\large{16pt}{18}}
\makeatother

\addto\captionsenglish{\renewcommand{\figurename}{Figure}}

% \usepackage{xstring}
% \usepackage{etoolbox}
% \usepackage{caption}

% \captionsetup{labelfont=bf,tableposition=top}

% \makeatletter
% \newcommand\formatlabel[1]{%
%     \noexpandarg
%     \IfSubStr{#1}{.}{%
%       \StrBefore{#1}{.}[\firstcaption]%
%       \StrBehind{#1}{.}[\secondcaption]%
%       \textbf{\firstcaption.} \secondcaption}{%
%       #1}%
%       }


% \patchcmd{\@caption}{#3}{\formatlabel{#3}}
% \makeatother

\renewcommand*{\Affilfont}{\normalsize\normalfont}
\renewcommand*{\Authfont}{\normalfont}


% referencing of unnumbered materials and methods
\newcounter{methods}
\renewcommand{\themethods}{Materials and methods}

% Track changes
%\usepackage[markup=underlined]{changes}
\makeatletter
\@namedef{Changes@AuthorColor}{magenta}
\colorlet{Changes@Color}{magenta}
\makeatother


%=====================================================================% Declare

\DeclareSIUnit\eur{\officialeuro}
\DeclareSIUnit\M{M}
\DeclareSIUnit\k{k}

% Widebar symbol
% \DeclareFontFamily{U}{mathx}{\hyphenchar\font45}
% \DeclareFontShape{U}{mathx}{m}{n}{<-> mathx10}{}
% \DeclareSymbolFont{mathx}{U}{mathx}{m}{n}
% \DeclareMathAccent{\widebar}{0}{mathx}{"73}

%=====================================================================% New commands (Macros)

% def
\def\sym#1{\ifmmode^{#1}\else\(^{#1}\)\fi}
\definecolor{darkgreen}{rgb}{0.0, 0.5, 0.0}

% new command
\newcommand{\smallsim}{\smallsym{\mathrel}{\sim}}
\newcommand{\specialcell}[2][c]{%
  \begin{tabular}[#1]{@{}l@{}}#2\end{tabular}}
\newcommand{\specialcellc}[2][c]{%
  \begin{tabular}[#1]{@{}c@{}}#2\end{tabular}}
\newcommand\ie{i.\,e.\xspace}
\newcommand\eg{e.\,g.\xspace}
\newcommand{\dd}[1][]{\mathrm{d}#1}
\newcommand{\BK}[1]{{\color{orange}{BK: #1}}}
\newcommand{\figletter}[1]{{{\fontfamily{\sfdefault}\selectfont \textbf{#1}}}}
\newcommand\TODO[1]{{\color{red}#1}}  
\newcommand{\FIX}[1]{{\color{darkgreen}#1}}  

% renewcommand
\renewcommand\theadfont{\bfseries}
\renewcommand\theadalign{lc}
\renewcommand\cellalign{tl}

\makeatletter

\newbox\@abstract%
\def\abstitle{\textbf{Abstract}}%
\renewenvironment{abstract}{
  \global\setbox\@abstract\vbox\bgroup%
   \noindent
}{%
   \egroup%
}%

\renewcommand*{\Affilfont}{\normalsize\normalfont}
\renewcommand*{\Authfont}{\normalfont}

\addto\captionsenglish{% Replace "english" with the language you use
  \renewcommand{\contentsname}{List of Texts}
}

\def\@maketitle{%
  \newpage
    {\raggedright\fontsize{18pt}{20pt}\selectfont \@title \par}%
    \vskip 0.5em%
    {\large
      \lineskip .5em%
      \begin{tabular}[t]{l}%
        \raggedright \normalsize\mdseries{\@author} %
      \end{tabular}\par}%
      \vskip 1em
%      \raggedright\Large\abstitle\par
%      \vskip 1em
%    {\unvbox\@abstract\par}%
    \par
  \vskip 0.5em
}
  
\makeatother


\renewcommand{\thesection}{Text \arabic{section}}
\usepackage{titlesec}
%\titleformat{\section}{\normalfont\Large\bfseries}{Text \thesection.~#1}{1em}{}
\renewcommand{\thefigure}{S\arabic{figure}}
\renewcommand{\thetable}{S\arabic{table}}

\begin{document}
\doublespacing
\nolinenumbers

\newcommand{\supp}{SI Appendix}

\title{\LARGE\singlespacing{\textbf{Supplmenetary Material} \\ \medskip
Molecular detection of SARS-CoV-2 and other respiratory viruses in saliva and bioaerosols}}

% long: Air cleaners and respiratory infections in schools: \\ A modeling study using epidemiological, environmental, and molecular data

% author list
\author[1,2]{Nicolas Banholzer}
\author[2,3]{Philipp Jent}
\author[2,4]{Pascal Bittel}
\author[1]{Kathrin Zürcher}
\author[4]{Lavinia Furrer}
\author[1]{Simon Bertschinger}
\author[5]{Ernest Weingartner}
\author[2,4]{Alban Ramette}
\author[1,6,7]{Matthias Egger}
\author[2,8]{Tina Hascher}
\author[1*,2]{Lukas Fenner}

\affil[1]{Institute of Social and Preventive Medicine, University of Bern, Bern, Switzerland}
\affil[2]{Multidisciplinary Center for Infectious Diseases, University of Bern, Bern, Switzerland}
\affil[3]{Department of Infectious Diseases, Inselspital, Bern University Hospital, University of Bern, Bern, Switzerland}
\affil[4]{Institute for Infectious Diseases, University of Bern, Bern, Switzerland}
\affil[5]{Institute for Sensors and Electronics, University of Applied Sciences and Arts Northwestern Switzerland, Windisch, Switzerland}
\affil[6]{Population Health Sciences, University of Bristol, Bristol, UK}
\affil[7]{Centre for Infectious Disease Epidemiology and Research, University of Cape Town, Cape Town, South Africa}
\affil[8]{Institute of Educational Science, University of Bern, Bern, Switzerland}

%\begin{abstract}\normalfont
%The supplementary material contains (1)~the detailed method, (2)~the simulation-based study, (3)~further descriptives, and (4)~the results from the sensitivity analysis.
%\end{abstract}

\flushbottom
\maketitle
\thispagestyle{empty}

%\newpage

\sloppy
\raggedbottom

\newpage

\appendix

\begin{table}[!htpb]
    \caption{Differences in the study settings of 2022 and 2023.}
    \label{tab:comp_study}
    \centering
    \renewcommand{\arraystretch}{1.5}
    \begin{tabular}{lp{6cm}p{6cm}}
        \toprule
        & \textbf{2022 (during pandemic)} & \textbf{2023 (after pandemic)} \\
        \midrule
        Participants & 2~classes from School~1 sharing one classroom (half-class teaching) and 1~class from School~2 & two classes from School~2 \\
        Duration & 7 weeks per class (January 16 to March 11, 2023) & 7 weeks per class (January 24 to March 26, 2023) \\
        Saliva testing & weekly (Wednesday) & bi-weekly (Tuesday and Thursday) \\
        Bioaerosol sampling & daily with BioSpot-VIVAS and Coriolis in both classrooms & daily with BioSpot-VIVAS and Coriolis in one classroom and only Coriolis in the other classroom \\
        HEPA-filters sampling & once at the end of the study per filter and classroom & two times per filter and classroom \\
        Interventions & compulsory face mask wearing as mandated by the public health authorities at that time from the start of the study for a total of 6 study weeks ; and installation of portable air cleaners towards the end of the study in the study classrooms for a total of 4 study weeks & installation of portable air cleaners in the study classrooms using a cross-over design for a total of 7 weeks \\
        Ventilation conditions & passive window ventilation as per recommendations of the national public health authorities; in School~1 ventilation was additionally assisted by a CO$_2$ guided opener of a small window at the top & passive window ventilation solely at the discretion of the teachers \\
        \bottomrule
    \end{tabular}
\end{table}

\clearpage

\section{Model to estimate differences in airborne detection}\label{sec:model}

Our outcome is a binary variable $Y_{tcv}$ which is $1$ if any respiratory virus $v$ was detected in the air of classroom $c$ during study week $t$. We model this outcome with a Bayesian logistic regression model
\begin{align}
    \text{Bernoulli-Logit}(y|\mu) = \begin{cases}
        \text{logit}^{-1}(\mu) & \text{if }y=1\text{, and} \\
        1-\text{logit}^{-1}(\mu) & \text{if }y=0.
    \end{cases}
\end{align}
The parameter $\mu$ is related to the type of respiratory virus and differences in the study setting (see \Cref{tab:comp_study}) as follows
\begin{align}
    \mu_{tcv} = \log(\text{offset}_c) + \alpha + \beta_1\,\cdot\text{saliva}_{tcv} + \beta_2\,\cdot\text{CoV}_{tcv} + \beta_3\,\cdot\text{mask}_{tcv} + \beta_4\,\cdot\text{aircl}_{tcv} + 
\end{align}
where
\begin{itemize}
    \item \textbf{offset}: Since one classroom in 2022 was shared by two classes, we set the offset of this classroom to 2 (double odds for airborne detection in this classroom). Conversely, since one classroom in 2023 only had one instead of two sampling devices in place, we set the offset of this classroom to 0.5 (half odds for airborne detection in this classroom). 
    \item  \textbf{saliva}: Binary variable indicating if there was a positive saliva sample in the same week. Airborne detection is more likely if the same virus was found in student saliva.
    \item \textbf{CoV}: Binary variable indicating if virus $v$ in classroom $c$ in week $t$ was SARS-CoV-2. Parameter $\beta_2$ models the log odds of detecting any SARS-CoV-2 vs non-SARS-CoV-2 in bioaerosols. This is our parameter of interest.
    \item \textbf{masks}: Binary variable indicating if face mask wearing was mandated. Masks possibly reduce the probability of airborne detection.
    \item \textbf{aircl}: Binary variable indicating if portable air cleaners were installed in the classroom. Air cleaners possibly reduce the probability of airborne detection.
    \item \textbf{CO}$_2$: Weekly average of the daily maximum CO$_2$ levels in each classroom. Higher CO$_2$ levels are a proxy of worse ventilation (outdoor air exchange) and possibly increase the probability of airborne detection.
\end{itemize}

We model the intercept with a diffuse prior
\begin{align}
    \alpha \sim \text{Student}(\nu=5, \mu = 0, \sigma = 2.5) ~.
\end{align}
Bioaerosols were more frequently found in weeks with a positive saliva sample. Therefore, we model $\beta_1$ with an informative prior
\begin{align}
    \beta_1 \sim \text{Normal}(\mu = 5/3, \sigma = s_y / s_{\text{saliva}}),
\end{align}
where 5 is the number of weeks with paired samples and 3 is the number of weeks with a positive bioaerosol but without a positive saliva sample. The prior's standard deviation is computed as the ratio of the standard deviation of the outcome $s_y$ and the variable $s_{\text{saliva}}$. \\
Our parameter of interest is modeled with a non-informative prior
\begin{align}
    \beta_2 \sim \text{Normal}(\mu = 0, \sigma = s_y / s_\text{CoV})~.
\end{align}
Masks and air cleaners possibly reduce the odds molecular airborne detection. To incorporate this prior believe into $\beta_3$ and $\beta_4$, we used an informative asymmetric Laplace prior 
\begin{align}
    \beta_3 &\sim \text{Asymmetric-Laplace}(\mu = -0.32, \sigma = 0.5\cdot s_y / s_{\text{masks}}, \tau = 0.75) \\
    \beta_4 &\sim \text{Asymmetric-Laplace}(\mu = -0.27, \sigma = 0.5\cdot s_y / s_{\text{aircl}}, \tau = 0.75),
\end{align}
which has a 90\% prior probability for a negative intervention effect and is skewed towards smaller effects. \\
Similarly, we incorporate our prior belief that higher CO$_2$ levels possibly increase the odds of airborne detection with the converse prior for $\beta_5$ as
\begin{align}
    \beta_3 &\sim \text{Asymmetric-Laplace}(\mu = 0.21, \sigma = 0.5\cdot s_y / s_{\text{masks}}, \tau = 0.25)~.
\end{align}
To summarize, we estimate the odd ratio for airborne detection of SARS-CoV-2, adjusting for differences in the setup of the classrooms (offset), positive saliva sample in the same week (saliva), the effects of interventions (masks and aircl), and differences in ventilation conditions (CO$_2$). We use informative priors because the sample size is small and insufficient to identify the effects of the confounders. 

\clearpage

\end{document}