\PassOptionsToPackage{top=3cm,left=3cm,right=3cm,bottom=3cm}{geometry}
\documentclass[fleqn,11pt]{wlscirep_supp}

\usepackage[]{minitoc}
\mtcsetdepth{secttoc}{3}
\setcounter{secnumdepth}{2}
\setcounter{tocdepth}{2}
\mtcsettitle{secttoc}{}


\usepackage[utf8]{inputenc}
\usepackage[T1]{fontenc}
\usepackage[english]{babel}
%\usepackage[top=3cm,left=3cm,right=3cm,bottom=3cm]{geometry}% by courtesy of Mico
\usepackage{lmodern}
\usepackage{bbm}
\usepackage{graphicx}
\usepackage{epstopdf}
\usepackage{colortbl}
\usepackage{siunitx}
\sisetup{
  detect-all,
  detect-weight=true,
  detect-family=true,
  mode=text,
%   detect-inline-family=math,
  group-separator={,},
%   group-minimum-digits={3}			
}
\usepackage{rotating}
\usepackage{tabularx}
\usepackage{tabu}
\usepackage{authblk}
\usepackage{mathtools}
\usepackage{overpic}
\usepackage{url}
\usepackage{tikz}
\usetikzlibrary{positioning}
\usetikzlibrary{arrows}
\usetikzlibrary{fit}
\usepackage{multirow}
\usepackage{float}
\usepackage[normalem]{ulem}
\usepackage{bm}
\usepackage{enumerate}
\usepackage[absolute,overlay%,showboxes
                        ]{textpos}
% \usepackage{caption}
\usepackage[font=small,labelfont=bf,justification=justified]{caption}
\usepackage{subcaption}
\usepackage{xspace}
\usepackage[colorinlistoftodos]{todonotes}
\usepackage{placeins}
\usepackage{makecell, booktabs}
\usepackage{eqparbox}
\usepackage{rotating}
\usepackage{graphicx}
\usepackage{xspace}
\usepackage{setspace}
%\usepackage{comment}
\usepackage[resetlabels,labeled]{multibib}
\newcites{Supp}{References}
\usepackage[sort&compress]{cleveref}
\Crefname{appendix}{Supplement}{Supplements}

%\usepackage{mathabx}

% Table formatting packages
\usepackage{dcolumn} % align decimal points in tables
\newcolumntype{d}[1]{D{.}{.}{#1}}
\usepackage{booktabs} 
\usepackage[flushleft]{threeparttable}
\usepackage{siunitx} % align on decimal point in tables
\usepackage{lineno}
\usepackage{etoolbox}

\usepackage{lscape}
\usepackage{longtable}
\usepackage{arydshln}

% math
\usepackage{amsmath,amsfonts,amssymb}
% additional math symbols
\DeclareFontFamily{U}{mathb}{}
\DeclareFontShape{U}{mathb}{m}{n}{
  <-5.5> mathb5
  <5.5-6.5> mathb6
  <6.5-7.5> mathb7
  <7.5-8.5> mathb8
  <8.5-9.5> mathb9
  <9.5-11.5> mathb10
  <11.5-> mathb12
}{}
\DeclareSymbolFont{mathb}{U}{mathb}{m}{n}
\DeclareMathSymbol{\ulsh}{3}{mathb}{"E8}
\DeclareMathSymbol{\ursh}{3}{mathb}{"E9}
\DeclareMathSymbol{\dlsh}{3}{mathb}{"EA}
\DeclareMathSymbol{\drsh}{3}{mathb}{"EB}

%% Patch 'normal' math environments:
\newcommand*\linenomathpatch[1]{%
  \cspreto{#1}{\linenomath}%
  \cspreto{#1*}{\linenomath}%
  \csappto{end#1}{\endlinenomath}%
  \csappto{end#1*}{\endlinenomath}%
}

\linenomathpatch{equation}
\linenomathpatch{gather}
\linenomathpatch{multline}
\linenomathpatch{align}
\linenomathpatch{alignat}
\linenomathpatch{flalign}

\linenumbers

% PLOS formatting
\makeatletter %only needed in preamble
\renewcommand\Large{\@setfontsize\Large{18pt}{18}}
\renewcommand\large{\@setfontsize\large{16pt}{18}}
\makeatother

\addto\captionsenglish{\renewcommand{\figurename}{Fig }}

% \usepackage{xstring}
% \usepackage{etoolbox}
% \usepackage{caption}

% \captionsetup{labelfont=bf,tableposition=top}

% \makeatletter
% \newcommand\formatlabel[1]{%
%     \noexpandarg
%     \IfSubStr{#1}{.}{%
%       \StrBefore{#1}{.}[\firstcaption]%
%       \StrBehind{#1}{.}[\secondcaption]%
%       \textbf{\firstcaption.} \secondcaption}{%
%       #1}%
%       }


% \patchcmd{\@caption}{#3}{\formatlabel{#3}}
% \makeatother

\renewcommand*{\Affilfont}{\normalsize\normalfont}
\renewcommand*{\Authfont}{\normalfont}


% referencing of unnumbered materials and methods
\newcounter{methods}
\renewcommand{\themethods}{Materials and methods}

% Track changes
%\usepackage[markup=underlined]{changes}
\makeatletter
\@namedef{Changes@AuthorColor}{magenta}
\colorlet{Changes@Color}{magenta}
\makeatother


%=====================================================================% Declare

\DeclareSIUnit\eur{\officialeuro}
\DeclareSIUnit\M{M}
\DeclareSIUnit\k{k}

% Widebar symbol
% \DeclareFontFamily{U}{mathx}{\hyphenchar\font45}
% \DeclareFontShape{U}{mathx}{m}{n}{<-> mathx10}{}
% \DeclareSymbolFont{mathx}{U}{mathx}{m}{n}
% \DeclareMathAccent{\widebar}{0}{mathx}{"73}

%=====================================================================% New commands (Macros)

% def
\def\sym#1{\ifmmode^{#1}\else\(^{#1}\)\fi}
\definecolor{darkgreen}{rgb}{0.0, 0.5, 0.0}

% new command
\newcommand{\smallsim}{\smallsym{\mathrel}{\sim}}
\newcommand{\specialcell}[2][c]{%
  \begin{tabular}[#1]{@{}l@{}}#2\end{tabular}}
\newcommand{\specialcellc}[2][c]{%
  \begin{tabular}[#1]{@{}c@{}}#2\end{tabular}}
\newcommand\ie{i.\,e.\xspace}
\newcommand\eg{e.\,g.\xspace}
\newcommand{\dd}[1][]{\mathrm{d}#1}
\newcommand{\BK}[1]{{\color{orange}{BK: #1}}}
\newcommand{\figletter}[1]{{{\fontfamily{\sfdefault}\selectfont \textbf{#1}}}}
\newcommand\TODO[1]{{\color{red}#1}}  
\newcommand{\FIX}[1]{{\color{darkgreen}#1}}  

% renewcommand
\renewcommand\theadfont{\bfseries}
\renewcommand\theadalign{lc}
\renewcommand\cellalign{tl}

\makeatletter

\newbox\@abstract%
\def\abstitle{\textbf{Abstract}}%
\renewenvironment{abstract}{
  \global\setbox\@abstract\vbox\bgroup%
   \noindent
}{%
   \egroup%
}%

\renewcommand*{\Affilfont}{\normalsize\normalfont}
\renewcommand*{\Authfont}{\normalfont}

\addto\captionsenglish{% Replace "english" with the language you use
  \renewcommand{\contentsname}{List of Texts}
}

\def\@maketitle{%
  \newpage
    {\raggedright\fontsize{18pt}{20pt}\selectfont \@title \par}%
    \vskip 0.5em%
    {\large
      \lineskip .5em%
      \begin{tabular}[t]{l}%
        \raggedright \normalsize\mdseries{\@author} %
      \end{tabular}\par}%
      \vskip 1em
%      \raggedright\Large\abstitle\par
%      \vskip 1em
%    {\unvbox\@abstract\par}%
    \par
  \vskip 0.5em
}
  
\makeatother


%\renewcommand{\thesection}{Text \arabic{section}}
\usepackage{titlesec}
\titleformat{\section}{\normalfont\Large\bfseries}{Text \thesection.~#1}{1em}{}
\renewcommand{\thefigure}{\Alph{figure}}
\renewcommand{\thetable}{\Alph{table}}

\begin{document}
\doublespacing
\nolinenumbers

\newcommand{\supp}{SI Appendix}

\title{\LARGE\singlespacing{\textbf{S1 Appendix} \\ \medskip
Transmission of airborne respiratory infections with and without air cleaners in a Swiss school during non-pandemic conditions: A modeling study of epidemiological, environmental, and molecular data}}

% author list
\author[1$\ddag$]{Nicolas Banholzer}
\author[1$\ddag$]{Kathrin Z\"urcher}
\author[2]{Philipp Jent}
\author[3]{Pascal Bittel}
\author[3]{Lavinia Furrer}
\author[1]{Matthias Egger}
\author[4]{Tina Hascher}
\author[1*]{Lukas Fenner}

\affil[1]{Institute of Social and Preventive Medicine, University of Bern, Bern, Switzerland}
\affil[2]{Department of Infectious Diseases, Inselspital, Bern University Hospital, University of Bern, Bern, Switzerland}
\affil[3]{Institute of Infectious Diseases, University of Bern, Bern, Switzerland}
\affil[4]{Institute of Educational Science, University of Bern, Bern, Switzerland}

\affil[*]{Corresponding author: lukas.fenner@ispm.unibe.ch }

\affil[$\ddag$]{These authors contributed equally to this work.}

%\begin{abstract}\normalfont
%The supplementary material contains (1)~the detailed method, (2)~the simulation-based study, (3)~further descriptives, and (4)~the results from the sensitivity analysis.
%\end{abstract}

\flushbottom
\maketitle
\thispagestyle{empty}

%\newpage

\sloppy
\raggedbottom

\newpage

\appendix

\tableofcontents

\listoffigures

\listoftables

%\listoffigures
%\listoftables

\newpage

\section{Epidemiological line list data}\label{sec:case-data}

Table~\ref{tab:epi-data-line-list} shows the line list data for respiratory cases based on epidemiological data. For each class and case, it shows the date when the student was absent, s/he reported symptoms, and returned to school. Recall that a case of respiratory infection was defined as an absence where the student reported a sickness with at least one of the following symptoms: fever, coughing, tiredness, loss of test or smell, sore throat, headache, aches and pains, diarrhea, difficulty breathing or shortness of breath, stomach.

{\footnotesize\begin{longtable}{l l l l}
    \caption[Line list of respiratory cases over the study period]{Line list of respiratory cases over the study period.}\label{tab:epi-data-line-list} \\
    \toprule
    Class & Date of absence & Date of symptom onset & Date of return \\
    \midrule
    \input ../../results/epi-data/line-list-data
    \bottomrule
\end{longtable}}

\clearpage

\section{Molecular line list data}\label{sec:mol-data}

Table~\ref{tab:mol-data-line-list} shows the line list data for the laboratory test results from human saliva samples. For each class and test, it shows the date when the test was taken, the test result and (if positive) the pathogen that was detected. Recall that all students in class were sampled twice per week every Tuesday and Thursday. 

{\footnotesize\begin{longtable}{l l l l l}
    \caption[Line list of molecular test results over the study period]{Line list of respiratory cases over the study period. Pathogens: Influenza B (Flu B), rhinovirus (HRV), adenovirus (AdV), SARS-CoV-2 (CoV), metapneumovirus (MPV), parainfluenza virus (PIV).}\label{tab:mol-data-line-list} \\
    \toprule
    Class & Date of test & Weekday & Test result & Pathogen \\
    \midrule
    \input ../../results/mol-data/line-list-data
    \bottomrule
\end{longtable}}

\clearpage

\section{Modeling relative risk of infection}\label{sec:transmission-model}

\subsection{Overall approach}

The overall aim is to estimate the effects of air cleaners on the daily number of respiratory infections. The latter is unobserved/latent and inferred from the daily number of respiratory cases (absences related to respiratory infections by date of symptom onset), considering the delay from infection to symptom onset (\ie the incubation period). The effect of air cleaners is estimated with a Bayesian approach, which requires the specification of prior distributions for all model parameters.


\subsection{Notation}

\begin{tabular}{ll} 
$j$  & class \\
$t$ & days since start of study period \\
$C_{jt}$  & number of  new cases among students in class $j$ at day $t$ (observed) \\
$I_{jt}$  & number of new infections in class $j$ at day $t$ (unobserved)  \\
$F_{jt}$  & number of new infections in class $j$ before day $t$ (unobserved)  \\
$N_{jt}$  & cumulative number of infections in class $j$ until day $t$ (unobserved)  \\
\end{tabular}  

\subsection{Relating the number of respiratory cases to the number of infections}

The number of respiratory cases $C$ in class $j$ at time $t$ is modeled with a Negative Binomial distribution   
\begin{align*}
    C_{jt} \sim \text{Negative-Binomial}(\mu_{jt},\phi),
\end{align*}
where $\mu_{jt}$ is the expected number of new cases and $\phi$ is the parameter modeling over-dispersion. The expected number of new cases is the weighted sum of the number of new infections $I_{jt}$ in the  previous days
\begin{align*}
    \mu_{jt} = \sum_{s<t}I_{js} \cdot p_{\text{IN}}(t-s),
\end{align*}
where $p_{\text{IN}}(t-s)$ denotes the probability distribution of the incubation period. 

\subsection{Relating the number of new infections to the presence of interventions}

The  number of new infections is related to the presence of air cleaners using a log-link
\begin{align*}
    \log I_{jt} &= \log F_{jt} - \log N_{jt} + \beta_0 + \beta_1 \cdot \text{AirCleaner}_{jt} + \beta_2 \cdot \text{Class}_t + \beta_3 \cdot \text{Students}_{jt} \\
    &\hphantom{= }+ \beta_4 \cdot \text{Ventilation}_{jt} + \beta_5 \cdot \text{CoV}_t + \beta_6 \cdot \text{ILI}_t,
\end{align*}
where $F_{jt} = \sum_{t-7}^{t-1} I_{js}$ is the number of infections in the previous seven days (first model offset; a proxy for the number of contagious students), $N_{jt} = \sum_{s<t} I_{js}$ is the cumulative number of infections (second model offset; the inverse is a proxy for the number of susceptible students), $\beta_0$ is the rate of new infections without air cleaners (model intercept), and $\beta_1$ is the effect of air cleaners. The effect of air cleaners is adjusted for class-specific effects, the number of students in school, the air change rate, the proportion of positive tests for SARS-CoV-2 in the community, and the number of consultations for influenza-like illnesses in the community. 

Note that we deviated from the statistical analysis plan by using as model offset $\log F_{jt} - \log N_{jt}$ instead of just $\log N_{jt}$. This change improved the fit of our model. The reason is that the ratio $\log (F_{jt} / N_{jt})$ incorporates both ways by which new infections can naturally decrease over time, \ie a decrease in contagious students ($F_{jt}$) or a decrease in susceptible students ($1 / N_{jt}$).

\subsection{Specifying the distribution of the incubation period}

The pathogen of each respiratory infection could not be identified from the epidemiological data because the students never obtained a laboratory test result. As a consequence, different incubation periods need to be considered in $p_{\text{IN}}$, reflecting a combination of the pathogen-specific incubation periods. The combination is determined based on the weekly proportion of positive saliva samples for each pathogen found in the molecular analysis. Formally, let $p_{\text{IN}}^{v}$ be the distribution for the incubation period of respiratory virus $v$ and let $pp_{vw}$ be the proportion of positive saliva samples in study week $w$, then each week the combined incubation period is computed as the weighted sum of the pathogen-specific incubation periods
\begin{align}
    p_{\text{IN}}^w(s) = \sum_{v} p_{\text{IN}}^{v}(s) \cdot pp_{vw} \quad \forall s.
\end{align}

The prior distributions for the pathogen-specific incubation periods are based on estimates published in the literature\cite{McAloon2020,Lessler2009LancetID}. The distributions are shown in \Cref{fig:prior-pin}. Since we could not obtain prior estimates for the incubation period of metapneumovirus (MPV) from the literature, we instead formed a distribution by using the equally weighted average of the parameters from the other distributions. 

We estimate the pathogen-specific distributions from our data as part of fitting the overall model. Furthermore, note that $p_\text{IN}$ is discretized via $p_\text{IN}(s) = \int_0^{0.5} p_\text{IN}(\tau) \;\text{d}\tau$ for $s = 0$ and $p_\text{IN}(s) = \int_{s-0.5}^{s+0.5} p_\text{IN}(\tau) \;\text{d}\tau$ for $s > 0$, where $p_\text{IN}(\tau) \sim \textrm{Lognormal}(\mu, \sigma)$ is the density of the lognormal distribution. 

\begin{figure}[!htpb]
    \centering
    \includegraphics{../../results/epi-data/incubation-periods.pdf}
    \caption[Choices of priors for the incubation periods]{Choices of priors for the distribution of the pathogen-specific incubation periods. Pathogens: Influenza B (Flu B), rhinovirus (HRV), adenovirus (AdV), SARS-CoV-2 (CoV), metapneumovirus (MPV), parainfluenza virus (PIV).}
    \label{fig:prior-pin}
\end{figure}

\subsection{Adjusting for under-reporting of cases on weekends}

Despite recording cases by date of symptom onset, we recorded a higher proportion of cases on Mondays than on weekends, suggesting recall bias and under-reporting of cases on weekends. To consider weekday effects in the reporting of cases, we re-weight the expected number of cases each week as follows. Let $k \in $(1: Saturday, 2: Sunday, 3: Monday, $\dots$, 7: Friday) denote the weekday with the week starting on Saturday. The re-weighted expected number of cases $\Tilde{\mu}$ (class and day indexes omitted) are computed as
\begin{align*}
    \Tilde{\mu}_{k} &= \mu_k \cdot \nu_k \cdot \left(\frac{\sum_k \mu_k}{\sum_k \nu_k \cdot \mu_k}\right)\\
    \sum_k \nu_k &= 1,
\end{align*}
where $\nu_k$ is the weight for weekday $k$. These weights are modeled with a Dirichlet prior
\begin{align*}
    \nu &\sim \text{Dirichlet}(c) \\
    c_k &= \sum_j \sum_t C_{jt} \mathbb{I}_{\text{Weekday}(t)=k}
\end{align*}
where $c_k$ is the total number of cases reported for weekday $k$ and $\mathbb{I}$ is a binary indicator. 

\subsection{Modeling school-free days}

Infections may have occurred during the week of vacation that falls into the study period. The expected number of infections and cases are computed during vacation, but vacation days are not modeled (\ie not incorporated into the model likelihood). In addition, we assume lower transmission of respiratory infections on days without school (weekends and vacations). We incorporate our prior belief into the model intercept $\beta_0$  
\begin{align*}
    \beta_0 &= \alpha + \omega \cdot \text{NoSchool},
\end{align*}
where $\alpha$ is the rate of new infections on school days and $\alpha + \omega$ is the rate on days without school. We model $\omega$ with an informative prior for a 10\% decrease in new infections on school-free days
\begin{align*}
    \omega &\sim \text{Normal}(\log 1.1, 0.05)~.
\end{align*}

\subsection{Seeding infections before study start}

Cases in the first week of the study could indicate infections before the study commenced. We will therefore seed our model $2 \cdot \text{m}\,$days before the study start, where $\text{m}$ is the average incubation period of the pathogen with the largest incubation period (\ie adenovirus). The number of infections before the study start will be modeled with an exponential prior
\begin{align*}
    I_{jt} \sim \text{Exponential}(1) \quad t = -2\cdot\text{m}+1, \dots, 0~. 
\end{align*}

Note that we deviated from the statistical analysis plan by changing the parameter of the Exponential distribution from $\lambda = 2 \cdot m$ to $\lambda = 1$. After inspecting the model fit, we realized that the model could not adequately fit the number of new cases at the start of study, because the expected number of new infections in the seeding period $\mu = 1 / \lambda = 1 / (2 \cdot m)$ were too low. After increasing the expected number of new infections in the seeding period to $\mu = 1$ the model could adequately capture new respiratory cases observed at the study start.

\subsection{Priors for modeling parameters}

We use weakly informative priors for all modeling parameters following recommendations on the choice of priors \cite{Gelman2008StatMed,Gelman2008StatAnnals,Gelman2020RegOther,Stan2020Priors,Gabry2023Priors}. The continuous adjustment variables are standardized to have zero mean and a standard deviation of 0.5. 
\begin{align*}
    \alpha &\sim \text{Student-t}_5(0, 10) \\
    \beta_2, \dots, \beta_6 &\sim \text{Student-t}_5(0, 2.5) \\
    \frac{1}{\sqrt{\phi}} &\sim \text{Half-Normal}(0,1),
\end{align*}
where Student-t$_5$ is a Student-t distribution with 5 degrees of freedom. 

\clearpage

\section{Modeling positivity rate of human saliva samples}\label{sec:multinomial-model}

\section{Modeling changes in particle concentrations}\label{sec:env-regression-model}

The change in aerosol number concentration $CN_{jt}$ in class $j$ at day $t$ (analogously for particle mass concentrations $PM_{jt}$) is estimated using Bayesian log-linear regression models, \ie
\begin{align}
    \log\,CN_{jt} = \alpha + \beta\,\textrm{School}_j + \bm{\gamma}\,\textrm{Weekday}_t + \theta_M\,M_{jt} + \theta_A\,A_{jt} + \zeta\,\log\,\textrm{N}_{jt} + \omega\,\log\,\textrm{AER}_{jt}~, 
\end{align}
where $\alpha$ is the log of the aerosol concentration without interventions, $\beta$ and $\bm{\gamma}$ are school and weekday effects, respectively, $\theta_M$ and $\theta_A$ are the effects of mask mandates ($M_{jt}$) and air cleaners ($A_{jt}$), respectively, $\zeta$ adjusts for the number of students in class ($\textrm{N}_{jt}$), and $\omega$ adjusts for the outdoor air exchange rate ($\textrm{AER}_{jt}$). The latter is computed from measured indoor CO$_2$ levels (see Section~\ref{sec:rav-computation}). The percent reduction in particle concentrations with interventions are quantified as $100 \times (\mathrm{e}^{-\theta} - 1)\,\%$.

\clearpage

\section{Detailed results for risk of infection model}\label{sec:detailed-redcap}

Table~\ref{tab:estimation-results} presents the posterior mean, credible intervals and model diagnostics for all model parameters. See the main paper for a discussion of the effects of interventions. Here we briefly discuss some of the additional model parameters.

The effective sample size (ESS) and the Gelman-Rubin convergence diagnostic ($\hat{R}$) indicate good estimation power. It further suggests that the Markov chains converged.

The estimate for $\tau$ indicates that there is variation in the effects of mask mandates between classes. However, the credible intervals of $\theta_j^M$ all include zero, indicating that there is only mild deviation of the class-specific estimates from their cross-class average estimate $\theta^M$.   

The mean estimate for the proportion of students being absent is negative, indicating that higher proportion of absences decrease the probability of getting infected. In contrast to that, the estimate for the reproduction number in the community is positive, indicating that higher community transmission increases the probability of getting infected. Both estimates are in line with our hypothesized effect, although it should be noted that both estimates have large credible intervals including zero.

The overdispersion parameter ($\phi$) cannot be precisely estimated, but the credible intervals indicate rather small overdispersion ($\phi \rightarrow \infty$).

The estimates for the logit of the probability of getting infected during school days without interventions ($\alpha$) are larger than the prior mean. This is due to the reduction in transmission from mask mandates. The weekend effect ($\omega$) is negative, but smaller than the mean of our informative prior, indicating that the likelihood suggests more comparable transmission on weekends.   

The posterior distribution of the parameters of the incubation period ($\mu^{p_{IN}}$ and $\sigma^{p_{IN}}$) are close to their prior, indicating that the data is not informative about the incubation period.  

\begin{table}[!htpb]
    \caption[Estimation results from transmission model]{Estimation results from transmission models across the 100 generated datasets for the number of new cases of COVID-19.}
    \label{tab:estimation-results}
    \footnotesize
    \centering
    \begin{tabular}{lrrrrr}
    \toprule
    Parameter & Mean & Lower 95\%-CrI & Upper 95\%-CrI & $\hat{R}$ & ESS \\
    \midrule
    \input ../../results/epi-data/estimation-results
    \bottomrule
    \multicolumn{6}{p{13cm}}{\scriptsize
        ESS is the effective sample size, \ie the number of independent MCMC samples with estimation power equivalent to the total number of autocorrelated samples\cite{Stan2022}, and $\hat{R}$ is the Gelman-Rubin convergence diagnostic \cite{Gelman1992}. Low ESS or $\hat{R}$ or $\hat{R}>1.10$ indicate bad convergence of the model \cite{Gelman2013}.}
    \end{tabular}
\end{table}

Our model accounts for the delay from infection to case confirmation and allows transmission to change only at the dates of interventions. It thus reflects overall trends in transmission during study conditions rather than day-to-day variation. To evaluate how well our model fits these trends, Fig.~\ref{fig:coverage} compares the estimated number of cases from the probabilistic simulation with the estimated number of new cases from the transmission model. Overall the estimates are in good agreement. The 95\%-CrI of the model-based estimates includes the 95\%-quantile of the simulation-based estimates. 

\begin{figure}[!htpb]
    \centering
    \includegraphics{../../results/epi-data/model-fit.pdf}
    \caption[Model- and simulation-based estimates of the number of COVID-19 cases]{Estimated (posterior mean as dot and 95\%-CrI as line) number of cases from our probabilistic simulation (blue) and transmission model (black).}
    \label{fig:coverage}
\end{figure}

\clearpage

\section{Detailed results for positivity rate in saliva samples}
\label{sec:detailed-molecular}

\begin{figure}[!htb]
\centering
    \includegraphics[width=10cm]{../../results/mol-data/model-results.pdf}
    \caption[Boxplot of environmental variables by school and study condition]{Boxplot for the daily average values of each environmental variable by school and study condition.}
    \label{fig:mol-estimation-results-sensitivity}
\end{figure}

\clearpage

\section{Detailed results for changes in particle concentrations}\label{sec:detailed-palas}

In the main paper, we showed results for changes in particle concentrations by study condition and summarized them across schools. In Fig.~\ref{fig:palas-supp-descriptives}, we present additional results disaggregated by schools and including other environmental variables. Furthermore, numerical estimation results for the reduction in particle concentrations with interventions are shown in Table~\ref{tab:palas-est-results}

\begin{figure}[!htb]
\centering
    \includegraphics[width=\linewidth]{../../results/env-data/otherVars-boxplot.pdf}
    \caption[Boxplot of environmental variables by school and study condition]{Boxplot for the daily average values of each environmental variable by school and study condition.}
    \label{fig:env-descriptives-other-vars}
\end{figure}

\begin{table}[!htpb]
    \caption[Estimated reduction in aerosol and particle concentrations with interventions]{Estimated reduction in aerosol number (CN) and particle mass (PM) concentrations with interventions (posterior mean and upper and lower estimate from the 95\%-CrI).}
    \label{tab:palas-est-results}
    \centering
    \footnotesize
    \begin{tabular}{l r r r}
    \toprule
    Variable & Mean & Lower & Upper \\
    \midrule
    \input ../../results/env-data/estimation-results-table.tex
    \bottomrule
    \end{tabular}
    
\end{table}

\clearpage

\bibliography{references.bib}

\end{document}