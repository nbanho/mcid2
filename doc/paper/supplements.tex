\PassOptionsToPackage{top=3cm,left=3cm,right=3cm,bottom=3cm}{geometry}
\documentclass[fleqn,11pt]{wlscirep_supp}

\usepackage[]{minitoc}
\mtcsetdepth{secttoc}{3}
\setcounter{secnumdepth}{2}
\setcounter{tocdepth}{2}
\mtcsettitle{secttoc}{}


\usepackage[utf8]{inputenc}
\usepackage[T1]{fontenc}
\usepackage[english]{babel}
%\usepackage[top=3cm,left=3cm,right=3cm,bottom=3cm]{geometry}% by courtesy of Mico
\usepackage{lmodern}
\usepackage{bbm}
\usepackage{graphicx}
\usepackage{epstopdf}
\usepackage{colortbl}
\usepackage{siunitx}
\sisetup{
  detect-all,
  detect-weight=true,
  detect-family=true,
  mode=text,
%   detect-inline-family=math,
  group-separator={,},
%   group-minimum-digits={3}			
}
\usepackage{rotating}
\usepackage{tabularx}
\usepackage{tabu}
\usepackage{authblk}
\usepackage{mathtools}
\usepackage{overpic}
\usepackage{url}
\usepackage{tikz}
\usetikzlibrary{positioning}
\usetikzlibrary{arrows}
\usetikzlibrary{fit}
\usepackage{multirow}
\usepackage{float}
\usepackage[normalem]{ulem}
\usepackage{bm}
\usepackage{enumerate}
\usepackage[absolute,overlay%,showboxes
                        ]{textpos}
% \usepackage{caption}
\usepackage[font=small,labelfont=bf,justification=justified]{caption}
\usepackage{subcaption}
\usepackage{xspace}
\usepackage[colorinlistoftodos]{todonotes}
\usepackage{placeins}
\usepackage{makecell, booktabs}
\usepackage{eqparbox}
\usepackage{rotating}
\usepackage{graphicx}
\usepackage{xspace}
\usepackage{setspace}
%\usepackage{comment}
\usepackage[resetlabels,labeled]{multibib}
\newcites{Supp}{References}
\usepackage[sort&compress]{cleveref}
\Crefname{appendix}{Supplement}{Supplements}

%\usepackage{mathabx}

% Table formatting packages
\usepackage{dcolumn} % align decimal points in tables
\newcolumntype{d}[1]{D{.}{.}{#1}}
\usepackage{booktabs} 
\usepackage[flushleft]{threeparttable}
\usepackage{siunitx} % align on decimal point in tables
\usepackage{lineno}
\usepackage{etoolbox}

\usepackage{lscape}
\usepackage{longtable}
\usepackage{arydshln}

% math
\usepackage{amsmath,amsfonts,amssymb}
% additional math symbols
\DeclareFontFamily{U}{mathb}{}
\DeclareFontShape{U}{mathb}{m}{n}{
  <-5.5> mathb5
  <5.5-6.5> mathb6
  <6.5-7.5> mathb7
  <7.5-8.5> mathb8
  <8.5-9.5> mathb9
  <9.5-11.5> mathb10
  <11.5-> mathb12
}{}
\DeclareSymbolFont{mathb}{U}{mathb}{m}{n}
\DeclareMathSymbol{\ulsh}{3}{mathb}{"E8}
\DeclareMathSymbol{\ursh}{3}{mathb}{"E9}
\DeclareMathSymbol{\dlsh}{3}{mathb}{"EA}
\DeclareMathSymbol{\drsh}{3}{mathb}{"EB}

%% Patch 'normal' math environments:
\newcommand*\linenomathpatch[1]{%
  \cspreto{#1}{\linenomath}%
  \cspreto{#1*}{\linenomath}%
  \csappto{end#1}{\endlinenomath}%
  \csappto{end#1*}{\endlinenomath}%
}

\linenomathpatch{equation}
\linenomathpatch{gather}
\linenomathpatch{multline}
\linenomathpatch{align}
\linenomathpatch{alignat}
\linenomathpatch{flalign}

\linenumbers

% PLOS formatting
\makeatletter %only needed in preamble
\renewcommand\Large{\@setfontsize\Large{18pt}{18}}
\renewcommand\large{\@setfontsize\large{16pt}{18}}
\makeatother

\addto\captionsenglish{\renewcommand{\figurename}{Figure}}

% \usepackage{xstring}
% \usepackage{etoolbox}
% \usepackage{caption}

% \captionsetup{labelfont=bf,tableposition=top}

% \makeatletter
% \newcommand\formatlabel[1]{%
%     \noexpandarg
%     \IfSubStr{#1}{.}{%
%       \StrBefore{#1}{.}[\firstcaption]%
%       \StrBehind{#1}{.}[\secondcaption]%
%       \textbf{\firstcaption.} \secondcaption}{%
%       #1}%
%       }


% \patchcmd{\@caption}{#3}{\formatlabel{#3}}
% \makeatother

\renewcommand*{\Affilfont}{\normalsize\normalfont}
\renewcommand*{\Authfont}{\normalfont}


% referencing of unnumbered materials and methods
\newcounter{methods}
\renewcommand{\themethods}{Materials and methods}

% Track changes
%\usepackage[markup=underlined]{changes}
\makeatletter
\@namedef{Changes@AuthorColor}{magenta}
\colorlet{Changes@Color}{magenta}
\makeatother


%=====================================================================% Declare

\DeclareSIUnit\eur{\officialeuro}
\DeclareSIUnit\M{M}
\DeclareSIUnit\k{k}

% Widebar symbol
% \DeclareFontFamily{U}{mathx}{\hyphenchar\font45}
% \DeclareFontShape{U}{mathx}{m}{n}{<-> mathx10}{}
% \DeclareSymbolFont{mathx}{U}{mathx}{m}{n}
% \DeclareMathAccent{\widebar}{0}{mathx}{"73}

%=====================================================================% New commands (Macros)

% def
\def\sym#1{\ifmmode^{#1}\else\(^{#1}\)\fi}
\definecolor{darkgreen}{rgb}{0.0, 0.5, 0.0}

% new command
\newcommand{\smallsim}{\smallsym{\mathrel}{\sim}}
\newcommand{\specialcell}[2][c]{%
  \begin{tabular}[#1]{@{}l@{}}#2\end{tabular}}
\newcommand{\specialcellc}[2][c]{%
  \begin{tabular}[#1]{@{}c@{}}#2\end{tabular}}
\newcommand\ie{i.\,e.\xspace}
\newcommand\eg{e.\,g.\xspace}
\newcommand{\dd}[1][]{\mathrm{d}#1}
\newcommand{\BK}[1]{{\color{orange}{BK: #1}}}
\newcommand{\figletter}[1]{{{\fontfamily{\sfdefault}\selectfont \textbf{#1}}}}
\newcommand\TODO[1]{{\color{red}#1}}  
\newcommand{\FIX}[1]{{\color{darkgreen}#1}}  

% renewcommand
\renewcommand\theadfont{\bfseries}
\renewcommand\theadalign{lc}
\renewcommand\cellalign{tl}

\makeatletter

\newbox\@abstract%
\def\abstitle{\textbf{Abstract}}%
\renewenvironment{abstract}{
  \global\setbox\@abstract\vbox\bgroup%
   \noindent
}{%
   \egroup%
}%

\renewcommand*{\Affilfont}{\normalsize\normalfont}
\renewcommand*{\Authfont}{\normalfont}

\addto\captionsenglish{% Replace "english" with the language you use
  \renewcommand{\contentsname}{List of Texts}
}

\def\@maketitle{%
  \newpage
    {\raggedright\fontsize{18pt}{20pt}\selectfont \@title \par}%
    \vskip 0.5em%
    {\large
      \lineskip .5em%
      \begin{tabular}[t]{l}%
        \raggedright \normalsize\mdseries{\@author} %
      \end{tabular}\par}%
      \vskip 1em
%      \raggedright\Large\abstitle\par
%      \vskip 1em
%    {\unvbox\@abstract\par}%
    \par
  \vskip 0.5em
}
  
\makeatother


\renewcommand{\thesection}{Text \arabic{section}}
\usepackage{titlesec}
%\titleformat{\section}{\normalfont\Large\bfseries}{Text \thesection.~#1}{1em}{}
\renewcommand{\thefigure}{S\arabic{figure}}
\renewcommand{\thetable}{S\arabic{table}}

\begin{document}
\doublespacing
\nolinenumbers

\newcommand{\supp}{SI Appendix}

\title{\LARGE\singlespacing{\textbf{S1 Appendix} \\ \medskip
Transmission of respiratory viral infections with and without air cleaners in a Swiss school under non-pandemic conditions in 2023: A modeling study of epidemiological, environmental, and molecular data}}

% author list
\author[1$\ddag$,2]{Nicolas Banholzer}
\author[2,3]{Philipp Jent}
\author[2,4]{Pascal Bittel}
\author[4]{Lavinia Furrer}
\author[4]{Alban Ramette}
\author[4]{Ronald Dijkman}
\author[5]{Jenna Kelly}
\author[1]{Kathrin Zürcher}
\author[1]{Matthias Egger}
\author[2,6]{Tina Hascher}
\author[1*,2]{Lukas Fenner}

\affil[1]{Institute of Social and Preventive Medicine, University of Bern, Bern, Switzerland}
\affil[2]{Multidisciplinary Center for Infectious Diseases, University of Bern, Bern, Switzerland}
\affil[3]{Department of Infectious Diseases, Inselspital, Bern University Hospital, University of Bern, Bern, Switzerland}
\affil[4]{Institute of Infectious Diseases, University of Bern, Bern, Switzerland}
\affil[5]{Interfaculty Bioinformatics Unit, University of Bern, Bern, Switzerland}
\affil[6]{Institute of Educational Science, University of Bern, Bern, Switzerland}

\affil[*]{Corresponding author: lukas.fenner@ispm.unibe.ch }

\affil[$\ddag$]{These authors contributed equally to this work.}

%\begin{abstract}\normalfont
%The supplementary material contains (1)~the detailed method, (2)~the simulation-based study, (3)~further descriptives, and (4)~the results from the sensitivity analysis.
%\end{abstract}

\flushbottom
\maketitle
\thispagestyle{empty}

%\newpage

\sloppy
\raggedbottom

\newpage

\appendix

\tableofcontents

\listoffigures

\listoftables

%\listoffigures
%\listoftables

\newpage

\section{Data collection}\label{sec:data-collection}

\Cref{tab:data} in \supp provides an overview of the types of data that were collected in each classroom.

\begin{table}[!htpb]
    \footnotesize
    \centering
    \caption{\textbf{Data collection}. Type of data collected, method/device, and frequency in the rooms of classes A and B.}
    \begin{tabular}{p{3.5cm}p{6cm} p{1cm} p{1cm} p{3cm}}
    \midrule
    Type of data & Method / device & Class A & Class B & Frequency \\
    \midrule
    \textbf{Molecular data} \\
    \midrule
    Bioaerosol sampling & Bioaerosol sampling devices: BioSpot-VIVAS (Aerosol Devices Inc., Ft. Collins, Colorado, United States) and Coriolis Micro Air (Bertin Instruments Montigny-le/Bretonneux, France) & x & (x) \newline only Coriolis & daily \\
    Saliva samples & Saliva samples & x & x & twice per week \\
    Swabs from air cleaners & Swab samples from Xiaomi HEPA filters (Xiaomi Mi Air Pro 70m2, Shenzhen, China) & x & x & after each intervention phase (both classes) and before the vacation (only class B) \\ 
    \midrule
    \textbf{Environmental data} \\
    \midrule
    CO$_2$, aerosol/particle concentrations, humidity, temperature & Air quality device: AQ Guard (Palas GmbH, Karlsruhe, Germany) & x & x & daily by minute \\
    Coughs & Sounds collected via microphones and coughs detected with a Machine Learning model \cite{Bertschinger2023CBMS} & x & x & daily by seconds \\
    \midrule
    \textbf{Epidemiological data} \\
    \midrule
    Absences & Survey & x & x & daily \\
    Student characteristics & Survey & x & x & at study start \\
    \bottomrule
    \end{tabular}
    \label{tab:data}
\end{table}

\clearpage

\section{Epidemiological line list data}\label{sec:case-data}

\Cref{tab:epi-data-line-list} shows the line list data for respiratory cases based on epidemiological data. For each class and case, it shows the date when the student was absent, s/he reported symptoms, and returned to school. Recall that a case of respiratory infection was defined as an absence where the student reported a sickness with at least one of the following symptoms: fever, coughing, tiredness, loss of test or smell, sore throat, headache, aches and pains, diarrhea, difficulty breathing or shortness of breath, stomach.

{\footnotesize\begin{longtable}{l l l l}
    \caption[Line list of respiratory cases over the study period]{Line list of respiratory cases over the study period.}\label{tab:epi-data-line-list} \\
    \toprule
    Class & Date of absence & Date of symptom onset & Date of return \\
    \midrule
    \input ../../results/epi-data/line-list-data
    \bottomrule
\end{longtable}}

\clearpage

\section{Molecular line list data}\label{sec:mol-data}

\Cref{tab:mol-data-line-list} shows the line list data for the laboratory test results from human saliva samples. For each class and test, it shows the date when the test was taken, the test result and (if positive) the virus that was detected. Recall that all students in class were sampled twice per week every Tuesday and Thursday. 

{\footnotesize\begin{longtable}{l l l l l}
    \caption[Line list of molecular test results over the study period]{Line list of respiratory cases over the study period. Viruses: Influenza B (Flu B), rhinovirus (HRV), adenovirus (AdV), SARS-CoV-2 (CoV), metapneumovirus (MPV), parainfluenza virus (PIV).}\label{tab:mol-data-line-list} \\
    \toprule
    Class & Date of test & Weekday & Test result & Virus \\
    \midrule
    \input ../../results/mol-data/line-list-data
    \bottomrule
\end{longtable}}

\clearpage

\section{Modeling relative risk of infection}\label{sec:transmission-model}

\subsection{Overall approach}

The overall aim is to estimate the effects of air cleaners on the daily number of respiratory infections. The latter is unobserved/latent and inferred from the daily number of respiratory cases (absences related to respiratory infections by date of symptom onset), considering the delay from infection to symptom onset (\ie the incubation period). The effect of air cleaners is estimated with a Bayesian approach, which requires the specification of prior distributions for all model parameters.


\subsection{Notation}

\begin{tabular}{ll} 
$j$  & class \\
$t$ & days since start of study period \\
$C_{jt}$  & number of  new cases among students in class $j$ at day $t$ (observed) \\
$I_{jt}$  & number of new infections in class $j$ at day $t$ (unobserved)  \\
$F_{jt}$  & number of new infections in class $j$ before day $t$ (unobserved)  \\
$N_{jt}$  & cumulative number of infections in class $j$ until day $t$ (unobserved)  \\
\end{tabular}  

\subsection{Relating the number of respiratory cases to the number of infections}

The number of respiratory cases $C$ in class $j$ at time $t$ is modeled with a Negative Binomial distribution   
\begin{align*}
    C_{jt} \sim \text{Negative-Binomial}(\mu_{jt},\phi),
\end{align*}
where $\mu_{jt}$ is the expected number of new cases and $\phi$ is the parameter modeling over-dispersion. The expected number of new cases is the weighted sum of the number of new infections $I_{jt}$ in the  previous days
\begin{align*}
    \mu_{jt} = \sum_{s<t}I_{js} \cdot p_{\text{IN}}(t-s),
\end{align*}
where $p_{\text{IN}}(t-s)$ denotes the probability distribution of the incubation period. 

\subsection{Relating the number of new infections to the presence of interventions}

The  number of new infections is related to the presence of air cleaners using a log-link
\begin{align*}
    \log I_{jt} &= \log F_{jt} - \log N_{jt} + \beta_0 + \beta_1 \cdot \text{AirCleaner}_{jt} + \beta_2 \cdot \text{Class}_t + \beta_3 \cdot \text{Students}_{jt} \\
    &\hphantom{= }+ \beta_4 \cdot \text{AirChangeRate}_{jt} + \beta_5 \cdot \text{CoV}_t + \beta_6 \cdot \text{ILI}_t,
\end{align*}
where $F_{jt} = \sum_{t-7}^{t-1} I_{js}$ is the number of infections in the previous seven days (first model offset; a proxy for the number of contagious students), $N_{jt} = \sum_{s<t} I_{js}$ is the cumulative number of infections (second model offset; the inverse is a proxy for the number of susceptible students), $\beta_0$ is the rate of new infections without air cleaners (model intercept), and $\beta_1$ is the effect of air cleaners. The effect of air cleaners is adjusted for class-specific effects, the number of students in school, the air change rate, the proportion of positive tests for SARS-CoV-2 in the community, and the number of consultations for influenza-like illnesses in the community. 

Note that we deviated from the statistical analysis plan by using as model offset $\log F_{jt} - \log N_{jt}$ instead of just $\log N_{jt}$. This change improved the fit of our model. The reason is that the ratio $\log (F_{jt} / N_{jt})$ incorporates both ways by which new infections can naturally decrease over time, \ie a decrease in contagious students ($F_{jt}$) or a decrease in susceptible students ($1 / N_{jt}$).

\subsection{Specifying the distribution of the incubation period}

The virus of each respiratory infection could not be identified from the epidemiological data because the students never obtained a laboratory test result. As a consequence, different incubation periods need to be considered in $p_{\text{IN}}$, reflecting a combination of the virus-specific incubation periods. The combination is determined based on the weekly proportion of positive saliva samples for each virus found in the molecular analysis. Formally, let $p_{\text{IN}}^{v}$ be the distribution for the incubation period of respiratory virus $v$ and let $pp_{vw}$ be the proportion of positive saliva samples in study week $w$, then each week the combined incubation period is computed as the weighted sum of the virus-specific incubation periods
\begin{align}
    p_{\text{IN}}^w(s) = \sum_{v} p_{\text{IN}}^{v}(s) \cdot pp_{vw} \quad \forall s.
\end{align}

The prior distributions for the virus-specific incubation periods are based on estimates published in the literature\cite{McAloon2020,Lessler2009LancetID}. The distributions are shown in \Cref{fig:prior-pin}. Since we could not obtain prior estimates for the incubation period of metapneumovirus (MPV) from the literature, we instead formed a distribution by using the equally weighted average of the parameters from the other distributions. 

We estimate the virus-specific distributions from our data as part of fitting the overall model. Furthermore, note that $p_\text{IN}$ is discretized via $p_\text{IN}(s) = \int_0^{0.5} p_\text{IN}(\tau) \;\text{d}\tau$ for $s = 0$ and $p_\text{IN}(s) = \int_{s-0.5}^{s+0.5} p_\text{IN}(\tau) \;\text{d}\tau$ for $s > 0$, where $p_\text{IN}(\tau) \sim \textrm{Lognormal}(\mu, \sigma)$ is the density of the lognormal distribution. 

\begin{figure}[!htpb]
    \centering
    \includegraphics{../../results/epi-data/incubation-periods.pdf}
    \caption[Choices of priors for the incubation periods]{Choices of priors for the distribution of the virus-specific incubation periods. Viruses: Influenza B (Flu B), rhinovirus (HRV), adenovirus (AdV), SARS-CoV-2 (CoV), metapneumovirus (MPV), parainfluenza virus (PIV).}
    \label{fig:prior-pin}
\end{figure}

\subsection{Adjusting for under-reporting of cases on weekends}

Despite recording cases by date of symptom onset, we recorded a higher proportion of cases on Mondays than on weekends, suggesting recall bias and under-reporting of cases on weekends. To consider weekday effects in the reporting of cases, we re-weight the expected number of cases each week as follows. Let $k \in $(1: Saturday, 2: Sunday, 3: Monday, $\dots$, 7: Friday) denote the weekday with the week starting on Saturday. The re-weighted expected number of cases $\Tilde{\mu}$ (class and day indexes omitted) are computed as
\begin{align*}
    \Tilde{\mu}_{k} &= \mu_k \cdot \vartheta_k \cdot \left(\frac{\sum_k \mu_k}{\sum_k \nu_k \cdot \mu_k}\right)\\
    \sum_k \vartheta_k &= 1,
\end{align*}
where $\nu_k$ is the weight for weekday $k$. These weights are modeled with a Dirichlet prior
\begin{align*}
    \vartheta &\sim \text{Dirichlet}(c) \\
    c_k &= \sum_j \sum_t C_{jt} \mathbb{I}_{\text{Weekday}(t)=k} \quad \forall k
\end{align*}
where $c_k$ is the total number of cases reported for weekday $k$ and $\mathbb{I}$ is a binary indicator. 

\subsection{Modeling school-free days}

Infections may have occurred during the week of vacation that falls into the study period. The expected number of infections and cases are computed during vacation, but vacation days are not modeled (\ie not incorporated into the model likelihood). In addition, we assume lower transmission of respiratory infections on days without school (weekends and vacations). We incorporate our prior belief into the model intercept $\beta_0$  
\begin{align*}
    \beta_0 &= \alpha + \omega \cdot \text{NoSchool},
\end{align*}
where $\alpha$ is the rate of new infections on school days and $\alpha + \omega$ is the rate on days without school. We model $\omega$ with an informative prior for a 10\% decrease in new infections on school-free days
\begin{align*}
    \omega &\sim \text{Normal}(\log 1.1, 0.05)~.
\end{align*}

\subsection{Seeding infections before study start}

Cases in the first week of the study could indicate infections before the study commenced. We will therefore seed our model $2 \cdot \text{m}\,$days before the study start, where $\text{m}$ is the average incubation period of the virus with the largest incubation period (\ie adenovirus). The number of infections before the study start will be modeled with an exponential prior
\begin{align*}
    I_{jt} \sim \text{Exponential}(1) \quad t = -2\cdot\text{m}+1, \dots, 0~. 
\end{align*}

Note that we deviated from the statistical analysis plan by changing the parameter of the Exponential distribution from $\lambda = 2 \cdot m$ to $\lambda = 1$. After inspecting the model fit, we realized that the model could not adequately fit the number of new cases at the start of study, because the expected number of new infections in the seeding period $\mu = 1 / \lambda = 1 / (2 \cdot m)$ were too low. After increasing the expected number of new infections in the seeding period to $\mu = 1$ the model could adequately capture new respiratory cases observed at the study start.

\subsection{Priors for modeling parameters}

We use weakly informative priors for all modeling parameters following recommendations on the choice of priors \cite{Gelman2008StatMed,Gelman2008StatAnnals,Gelman2020RegOther,Stan2020Priors,Gabry2023Priors}. The continuous adjustment variables are standardized to have zero mean and a standard deviation of 0.5. 
\begin{align*}
    \alpha &\sim \text{Student-t}_5(0, 10) \\
    \beta_2, \dots, \beta_6 &\sim \text{Student-t}_5(0, 2.5) \\
    \frac{1}{\sqrt{\phi}} &\sim \text{Half-Normal}(0,1),
\end{align*}
where Student-t$_5$ is a Student-t distribution with 5 degrees of freedom. 

\clearpage

\section{Modeling positivity rate of human saliva samples}\label{sec:multinomial-model}

Molecular analysis determined which saliva samples were positive for a respiratory virus. Let $p = 0, 1, \dots, P$ denote the virus, where $0$ refers to negative samples and $P$ is the number of different viruses detected over the study. The number of positive bioaerosol and saliva samples $y_p$ in samples of sizes $n$ is analyzed with a Multinomial logistic regression model
\begin{align*}
    y_0, y_1, \cdots, y_P | n, \beta_0, \beta_1, \beta_2, \beta_3 &\sim \text{Multinomial-Logit}(\theta_0, \theta_1, \dots \theta_P) \\
    \theta &= \text{softmax}(\mu) \\
    \mu &= \begin{cases}
                0, & \text{if } p=0, \\
                \beta_{0p} + \beta_{1} \cdot \text{AirCleaner} + \beta_2 \cdot \text{Class} + \beta_3 \cdot \text{Susceptibles}_p, & \text{if } p>0.
            \end{cases} \\
    \beta_{0p} &\sim \text{Normal}(0, 2.5) \quad \forall p \\
    \beta_1, \beta_2, \beta_3 &\sim \text{Student-t}_5\left(0, \frac{2.5}{s_{x}}\right) 
\end{align*}
where $\text{softmax}(\mu) = \exp(\mu)/\sum\exp(\mu)$ and $s_{x}$ is the empirical standard deviation of each input variable.  The negative test is set as the reference category. Overall variation in positive samples by virus will be modeled with $\beta_{0p}$ and the effect of air cleaners will be modeled with $\beta_1$. The effect of air cleaners is adjusted for class-specific effects and the decreasing number of susceptibles over time (for each virus computed as the number of students minus the total number of positive samples).

Note that we deviated from the statistical analysis plan by estimating $\beta_{0p}$ and $\beta_1$ without a hierarchical prior. First, there was considerable variation in the overall positivity rate of each virus, so that we decided to use unpooled intercepts $\beta_{0p}$ instead of partially pooled intercepts $\beta_{0[p]}$ for each virus. Second, there was insufficient variation in the data to inform partially pooled estimates $\beta_{1[p]}$ for the effects of air cleaners. Therefore, we decided to only estimate the average effect across viruses, \ie the completely pooled estimate $\beta_1$.  

\section{Modeling changes in particle concentrations}\label{sec:env-regression-model}

We only analyze particle concentration data of the time periods during which students were in the classroom. Particle concentrations $Y$ on day $t$ in class $j$ are summarized daily with the mean across these time periods. The change in aerosol number and particle matter mass concentrations is then estimated using Bayesian log-linear regression models
\begin{align*}
    \log Y | \beta_0, \beta_1, \dots, \beta_6 &\sim \text{Normal}(\mu,\sigma) \\
    \mu &= \beta_0 + \beta_1 \cdot \text{AirCleaner} + \beta_2 \cdot \text{Class} + \beta_3 \cdot \text{Weekday} + \beta_4 \cdot \text{Students} \\
    &\hphantom{= } + \beta_5 \cdot \text{AirChangeRate} + \beta_6 \cdot \text{Cases} \\
    \beta_0 &\sim \text{Student-t}_5(\mu_y, 2.5s_{y}) \\
    \beta_1, \dots, \beta_6 &\sim \text{Student-t}_5\left(0, 2.5\frac{s_{y}}{s_{x}}\right) \\
    \sigma &\sim \text{Exponential}\left(\frac{1}{s_{y}}\right)~.
\end{align*}
The effect of air cleaners is adjusted for class- and weekday-specific effects, the number of students in school, the air change rate, and the cumulative number of cases related to respiratory infections. A log transform is applied to all continuous input variables. 

\clearpage

\section{Modeling changes in cough frequency}\label{sec:aud-regression-model}

We analyze the daily number of coughs $Y$ with a Negative Binomial regression model
\begin{align*}
    Y | \beta_0, \beta_1, \dots, \beta_6 &\sim \text{Negative-Binomial}(\mu,\phi) \\
    \log \mu &= \log T + \beta_0 + \beta_1 \cdot \text{AirCleaner} + \beta_2 \cdot \text{Class} + \beta_3 \cdot \text{Weekday} + \beta_4 \cdot \text{Students} + \beta_5 \cdot \text{Ventilation} + \beta_6 \cdot \text{Cases} \\
    \beta_0 &\sim \text{Student-t}_5(0, 2.5) \\
    \beta_1, \dots, \beta_6 &\sim \text{Student-t}_5\left(0, \frac{2.5}{s_{x}}\right) \\
    \frac{1}{\sqrt{\phi}} &\sim \text{Half-Normal}(0,1),
\end{align*}
where $T$ is the daily duration that students were in the classroom and $\phi$ is the parameter modeling over-dispersion in count data. 

In addition, we analyze the association between the daily number of coughs $Y$ and the number of positive saliva test resutls for respiratory viruses with a Negative Binomial hierarchical regression model
\begin{align*}
    Y | \beta_0, \theta_{[v]}, \theta, \tau &\sim \text{Negative-Binomial}(\mu,\phi) \\
    \log \mu &= \log T + \beta_0 + \sum_v \theta_{[v]} \cdot \text{Saliva}_v \\
    \beta_0 &\sim \text{Student-t}_5(0, 2.5) \\
    \theta_{[\text{Flu~B}]}, \dots, \theta_{[\text{PIV}]} &\sim \text{Normal}\left(\theta, \tau\right) \\
    \theta &\sim \text{Student-t}_5\left(0, \frac{2.5}{s_{v}}\right) \\
    \tau &\sim \text{Half-Normal}\left(0, 1\right) \\
    \frac{1}{\sqrt{\phi}} &\sim \text{Half-Normal}(0,1),
\end{align*}
where $\theta_{[v]}$ is the partially pooled estimate for respiratory virus $v \in$ (Flu~B, HRV, AdV, CoV, MPV, PIV), $\tau$ estimates variation between viruses, and $s_v$ is the average standard deviation across the count variables of the respiratory viruses. 

\clearpage

\section{Detailed results for risk of infection model}\label{sec:detailed-redcap}

Table~\ref{tab:estimation-results} presents the posterior mean, credible intervals and model diagnostics for all model parameters. See the main paper for a discussion of the effects of interventions. Here we briefly discuss some of the additional model parameters.

The effective sample size (ESS) and the Gelman-Rubin convergence diagnostic ($\hat{R}$) indicate good estimation power. It further suggests that the Markov chains converged. 

The estimate for $\beta_2$ is positive, suggesting a higher risk of infection in class $B$, which is in line with the higher number of cases in this class. The estimate for $\beta_3$ is positive, suggesting a higher risk of infection as more students were in class. The estimate for $\beta_4$ is negative, suggesting a lower risk of infection with higher air change rates, in line with the intuition that the risk of infection is lower when the classroom is better ventilated. The estimate for $\beta_5$ and $\beta_6$ are both positive although only the former is distinguishable from zero, suggesting that higher transmission in the community was also associated with a higher risk of infection in the school. 

The overdispersion parameter ($\phi$) cannot be precisely estimated, but the credible interval suggests rather small overdispersion ($\phi \rightarrow \infty$). The school-free effect $\omega$, the weekday weights $\vartheta$, and the parameters for the distribution of the incubation period $\mu_{p_{\text{IN}}}$ and $\sigma_{p_{\text{IN}}}$ are all close to their priors, suggest that these parameters are barely informed by date but more by our prior assumptions.

\begin{table}[!htpb]
    \caption[Estimation results from infection risk model]{Estimation results from infection risk model for the number of new respiratory cases.}
    \label{tab:estimation-results}
    \footnotesize
    \centering
    \begin{tabular}{lrrrrr}
    \toprule
    Parameter & Mean & Lower 95\%-CrI & Upper 95\%-CrI & $\hat{R}$ & ESS \\
    \midrule
    \input ../../results/epi-data/estimation-results
    \bottomrule
    \multicolumn{6}{p{13cm}}{\scriptsize
        ESS is the effective sample size, \ie the number of independent MCMC samples with estimation power equivalent to the total number of autocorrelated samples\cite{Stan2022}, and $\hat{R}$ is the Gelman-Rubin convergence diagnostic \cite{Gelman1992}. Low ESS or $\hat{R}$ or $\hat{R}>1.10$ indicate bad convergence of the model \cite{Gelman2013}.}
    \end{tabular}
\end{table}

Our model accounts for the delay from infection to case confirmation and allows transmission to change only at the dates of interventions. It thus reflects overall trends in transmission during study conditions rather than day-to-day variation. To evaluate how well our model fits these trends, \Cref{fig:coverage} compares the model-estimated expected number of cases  with the observed number cases only on a weekly basis and across classes. Overall the estimates are in relatively good agreement. The 95\%-CrIs of the model-estimates always include the observed cases. 

\begin{figure}[!htpb]
    \centering
    \includegraphics{../../results/epi-data/model-fit.pdf}
    \caption[Comparison of model-estimates with observed case data]{Model-estimated number of new respiratory cases (posterior mean as red line and 50\%-, 80\%, and 95\%-CrIs as shaded areas) and the observed number of new respiratory cases (black line) across classes by study week.}
    \label{fig:coverage}
\end{figure}

\clearpage

\section{Detailed results for positivity rate in saliva samples}
\label{sec:detailed-molecular}

There can be unknown delay between when students were infected and when they tested positive. Therefore, we performed a sensitivity analysis where we lagged the molecular test results (number of positive saliva samples) by one test date (\ie from Tuesday to last Thursday and from Thursday to Tuesday) and one test week (\ie from Tuesday to last Tuesday and Thursday to last Thursday). The results are shown in \Cref{fig:mol-estimation-results-sensitivity}. The estimated effect of air cleaners remains indistinguishable from zero regardless of the considered lag.  

\begin{figure}[!htb]
\centering
    \includegraphics[width=10cm]{../../results/mol-data/model-results.pdf}
    \caption[Sensitivity analysis for the estimated effect of air cleaners on the positivity rate of saliva samples]{Sensitivity analysis for the estimated effect of air cleaners on the positivity rate of saliva samples. Shown is the adjusted relative risk (posterior mean as dot and 50\%-, 80\%, and 95\%-CrIs as lines) for air cleaners by lag of the molecular test results.}
    \label{fig:mol-estimation-results-sensitivity}
\end{figure}

\clearpage

\section{Detailed results for changes in particle concentrations}\label{sec:detailed-palas}

In the main paper, we compared particle concentrations between study conditions. \Cref{fig:env-descriptives-other-vars} compares additional environmental variables, \ie CO$_2$, air change rate, relative humidity, and temperature. There are no considerable differences between in any of the variables between study conditions. Furthermore, numerical estimation results for the reduction in particle concentrations with air cleaners are shown in \Cref{tab:palas-est-results}.

\begin{figure}[!htb]
\centering
    \includegraphics[width=\linewidth]{../../results/env-data/otherVars-boxplot.pdf}
    \caption[Boxplot of environmental variables by school and study condition]{Boxplot for the daily average values of each environmental variable by school and study condition.}
    \label{fig:env-descriptives-other-vars}
\end{figure}

\begin{table}[!htpb]
    \caption[Estimated reduction in aerosol and particle concentrations with interventions]{Estimated reduction in aerosol number (CN) and particle mass (PM) concentrations with interventions (posterior mean and upper and lower estimate from the 95\%-CrI).}
    \label{tab:palas-est-results}
    \centering
    \footnotesize
    \begin{tabular}{l r r r}
    \toprule
    Variable & Mean & Lower & Upper \\
    \midrule
    \input ../../results/env-data/estimation-results-table.tex
    \bottomrule
    \end{tabular}
    
\end{table}

\clearpage

\bibliography{references.bib}

\end{document}